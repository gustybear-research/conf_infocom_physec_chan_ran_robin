\section{Related Work}


Physical-layer security was pioneered by Wyner's work on the wiretap channel \cite{wyner1975wire}, which leverages the \textit{channel advantage} for legitimate receivers over degraded eavesdroppers to guarantee secure transmission over wireless channels. 
In \cite{wyner1975wire}, the rate of secret communications is characterized by \textit{secrecy capacity}, which is shown to be the difference in the capacity of the receiver and the eavesdropper. Following Wyner's work, numerous studies based on various channel models ranging from basic Gaussian channels to complex MIMO wiretap channels have been proposed later \cite{csiszar1978broadcast,leung1978gaussian,parada2005secrecy,li2007secret,gopala2008secrecy,KhistiSecureTransmissionMultiple2010,KhistiSecureTransmissionMultiple2010a}. In particular, Khisti et al. \cite{KhistiSecureTransmissionMultiple2010,KhistiSecureTransmissionMultiple2010a}
showed the secrecy capacity bounds in the large antenna limit with full channel state information (CSI) assumption. Their works reveal an important result   that the achievable secrecy capacity can be significantly affected by the number of antennas of the eavesdropper. 
% For instance, to block secret communication, Eve only needs three times as many antennas as transceivers have. 
However, since those theoretical works often make unrealistic assumptions such as channel advantage, full channel knowledge, or independent and identically channel distribution, they are rarely adopted to evaluate the secrecy of real-world schemes.

On the other hand, various practical physical-layer secret communication schemes  have been proposed.
One example is the friendly jamming approach.
Gollakota et al. prevented unauthorized commands from being transmitted to implantable medical devices (IMDs) in \cite{gollakota2011they}. They assume that the attacker equipped with MIMO is unable to separate the legitimate and jamming signal, due to the close proximity between the jammer and the data source. Similarly, Shen et al. \cite{shen2013ally} designed another jamming technique where jamming signals are controlled with secret keys, so that they are recoverable to authorized devices but unpredictably interfering to unauthorized ones. The jammer and the authorized device are very close to each other in both schemes, and this design is found as vulnerable by Tippenhauer et al. in \cite{tippenhauer2013limitations}. When an attacker tactfully places
her antenna array, the transmitted data signal can be recovered by exploiting the phase offsets between received signal components.
% Orthogonal blinding \cite{anand2012strobe} proposed by Anand et al. is another example of physical-layer security schemes. 
Artificial noise injection strategy \cite{negi2005secret,goel2008guaranteeing,liao2010qos,li2011safe} is another example  \cite{negi2005secret,goel2008guaranteeing,liao2010qos,li2011safe}, it has drawn significant attention by the security community since first proposed by Goel and Negi. However, it also relies on the unrealistic assumption that the statistics of the eavesdropper's channel are known to the transmitter. Argyraki et al.  in \cite{argyraki2013creating} proposed a cooperative jamming strategy for group secret agreement. By injecting  artificial noise through beamforming, a group of legitimate users are enable to create a shared secret, that the eavesdropper obtains very little information. However, this approach limits the number of antennas the eavesdropper possesses, which can be vulnerable to powerful eavesdroppers. On the other hand, Anand et al. proposed the orthogonal blinding scheme where no channel information about the eavesdropper is required \cite{anand2012strobe}.
To defend against a single-antenna eavesdropper, the transmitter injects artificial noise into channels orthogonal to the legitimate receiver's channel so that the original signal intended for the receiver cannot be recovered from the signal and noise mixture. However, when the eavesdropper has multiple antennas, by exploiting the known parts of the transmitted signal such as frame preambles, Schulz et al. \cite{schulz2014practical} successfully implemented a known-plaintext attack against orthogonal blinding. With normalized least mean square algorithms, an adaptive filter was trained to separate transmitted messages from artificial noise.

The root cause of the vulnerability in orthogonal blinding is that the channel is assumed to be stable during the whole transmission period, so that the attacker is able to gather enough plaintexts for filter training, and this flaw can be amended with channel randomization approach.
In the literature, the channel randomization approach has been used for key generation, message confidentiality, and integrity protection. Aono et al. \cite{aono2005wireless} proposed a key generation and agreement scheme that blocks the eavesdropper from generating the same key as transceivers by increasing the fluctuation of the wireless channel with a smart antenna. Hassanieh et al. \cite{hassanieh2015securing} presented a secret transmission scheme for RFIDs randomizing both modulation and channel by rotating several directional antennas at the transmitter. Different from this work, their scheme is only applicable to single-antenna transmitters and does not use pre-coding. To defend against active man-in-the-middle attacks, Hou et al. \cite{hou2015message} and Pan et al. \cite{pan2017message} randomized the wireless channel with a fan and a reconfigurable antenna respectively to prevent  online  signal cancellation.  All these works show that channel randomization approach can be a powerful tool to enhance physical-layer security. However, the studies are still preliminary and a comprehensive scheme that is MIMO-compatible and secure against multi-antenna attackers is lacking. 
