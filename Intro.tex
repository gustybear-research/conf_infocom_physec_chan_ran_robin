\section{Introduction}
The ever-expanding wireless technology is pushing the limit of the network security infrastructure. Many wireless devices need to secure the communication channels between each other without pre-shared security context. Orthogonal blinding based physical-layer security \cite{negi2005secret,goel2008guaranteeing,liao2010qos,li2011safe,anand2012strobe,argyraki2013creating} has been widely considered as a promising candidate to provide confidentiality during wireless transmission without \textit{a priori} key exchange. Instead of relying on pre-shared secrets, orthogonal blinding achieves secure communications by transmitting artificial noise into the null-space of the receiver's channel and  corrupting the eavesdropper's reception. Its practicality supersedes other theoretical physical-layer methods, such as zero-forcing beamforming,  which relies on knowledge about the eavesdropper's channel. Security analysis proves that it can asymptotically approach the secrecy rate of zero-force beamforming against single-antenna eavesdroppers. However, further studies show that orthogonal blinding is not effective against a multi-antenna eavesdropper, who has sufficient spatial dimensions to separate the message from the artificial noise. Schulz and Zheng et al. \cite{schulz2014practical, ZhengHighlyEfficientKnownPlaintext2015, zheng2016profiling} demonstrated that an eavesdropper may leverage the known or low entropy symbols in the transmission to quickly train a decoding filter and recovers the rest of the transmission, an attack equivalent to the known-plaintext attack in cryptanalysis.

The root of this vulnerability is due to the fact that the artificial noise only changes the quality of the receiving signal but not the state of the channel. Specifically, the noise injected by the transmitter (Alice) can lower the signal-to-noise ratio (SNR) of the eavesdropper's (Eve's) channel. But it cannot change the channel states between she and Eve or she and the legitimate receiver (Bob). This limitation opens up a window for the known-plaintext attack. Assuming the channel state remains ergodic with its coherent time. Due to the increasingly sophisticated digital modulation methods, Alice can transmit a sequence of tens or hundreds of symbols within such a short period. Although these symbols are buried deep under the artificial noise, a fraction of known symbols among them would allow Eve with multiple antennas to compute the channel state information (CSI), using a common MIMO technique known as least square (LS) channel estimation, which is robust against channel noise. Once Eve estimated the CSI, she may use it to equalize the channel and remove the artificial noise during the rest of the coherent period.

Follow this line of reasoning, there are two ways to defend against the known-plaintext attack, assuming Alice cannot avoid transmitting known symbols. She can limit the number of symbols to transmit within each coherent time period, which limits the communication throughput. Or she can reduce the coherent time to thwart the known-plaintext attack. However, the coherent time is an intrinsic condition that depends on the channel multipath and Doppler spread, both of which are not subject to the manipulation of transmitting content. Therefore, it would appear there are no cogent methods to defend against the known-plaintext attack.

However, in this paper, we challenge this no-win scenario and propose an orthogonal blinding based physical-layer security method immune to the known-plaintext attack: Channel-\textbf{R}andomized \textbf{O}rthogonal \textbf{B}l\textbf{in}ding (ROBin). ROBin leverages a pattern reconfigurable antenna to vary the channel state at a per symbol or per frame rate, resulting in an artificially created fast-changing wireless channel unsuitable for the known-plaintext attack, for which can be viewed as one of the proactive/dynamic defense (or moving target defense) mechanisms. To prevent the antenna reconfiguration from affecting Bob, we design a compressive sensing based algorithm for Alice to estimate the angle-of-departure (AoD) distribution of the multipath environment and predict the CSI for a given reconfigurable antenna pattern. Based on the predicted CSI, Alice can equalize the channel for Bob via digital pre-coding before transmitting. As a result, the main channel state appears stable  to Bob but randomly changing from Eve's perspective.

We formally analyze the security of ROBin, by comparing the mutual information between Alice's transmission and Eve's reception, assuming the channel state has the Markov property and Eve knows the symbols transmitted via historical antenna modes but not the current one. The analysis shows that Eve gains little advantage from knowing previous symbols (as channel randomization reduces the channel correlation  and makes  the current channel state more unpredictable). We implement the key components of ROBin; validate    our theoretical analysis with extensive simulation and real-world experiments. Empirical results show that our scheme can suppress Eve's attack success rate to the level of random guessing, even if she knows all the symbols transmitted through other modes.

